% !TEX TS-program = xelatex
% !TEX encoding = UTF-8 Unicode
\documentclass[10pt, a4paper]{article}
\usepackage{fontspec}
% Set up fonts
\usepackage{xunicode}
\usepackage{xltxtra}
\defaultfontfeatures{Mapping=tex-text}
\setromanfont [Ligatures={Common}, Numbers={OldStyle}]{Linux Libertine O}
% Layout
\usepackage[xetex]{geometry}
\geometry{a4paper, textheight=8.5in}

% ------------------------------------------------------------------ %
% DOCUMENT DATA                                                      %
% ------------------------------------------------------------------ %
\title{
  {\small Final Year Project Proposal:}\\
  {A Data-Driven Self-Awareness Model for Robotics Systems}}
\author{
  Jeremiah M. Via \\ 
  \small{Supervised by Nick Hawes}}
\date{7 October 2011}


% ------------------------------------------------------------------ %
% BEGIN DOCUMENT                                                     %
% ------------------------------------------------------------------ %

\begin{document}
\maketitle

\begin{abstract}
  Fault-detection in robotics systems is a difficult task. They are
  continually becoming more complex and thus subtle errors are hard to
  diagnose. Traditional approaches have relied on explicitly modeling
  component faults in code, but this technique does not scale to
  complex robots operating in dynamic environments. A new technique
  involves making the robot self-aware to the internal state of its
  various components. The aim of this project is to measure the
  efficacy of a probabilistic self-awareness model for robotics
  systems on two robotics middlewares, ROS and CAST, and if time
  allows, extend the model using a neural network, and then compare
  the two approaches. During this project, I will be collaborating
  with the creator of the original approach.
\end{abstract}


\section*{Description}

This is a project with a heavy research bias. It involves implementing
software to connect the CAST and ROS robotics middlewares to
state-of-the-art software for data-driven fault detection and then
measuring how well this fault-detection system performs on these
middlewares. It also involves extending this technique with a
different machine learning approach, neural networks. My hypothesis is
that neural networks should generalize better than the current
approach.

This software has multiple uses. It can be used during development as
a debugging tool for the developer. They can see if their software is
causing the robot to enter faulty states caused by any number of
things. More interestingly, this information can be fed back to the
robot's planning system. This is what truly creates that idea of
self-awareness. A robot which can determine if its components are not
running correctly can take action to overcome these faults, and is
thus much more robust. This is of critical importance when robots are
working around humans.

\subsection*{User Specification}

This software is intended for robotics researchers and possibly those
developing industrial robots. It will require a minimal understanding
of machine learning to successfully incorporate into a robotics
system. As such, this software will target expert computer users at
the cutting-edge of robotics. I hope to make the software as intuitive
as possible, but given that this is a machine learning system, it
might be difficult.

\subsection*{Functional Specification}

In order to measure the efficacy of the probabilistic self-awareness
model, a bridge will have to be written to connect the two middlewares
to the system which implements the probabilistic model. This is
absolutely essential. Additionally, I will have to create a bridge to
connect the system I write. 

If time allows, I hope to create a neural network which can perform
the task. It does not matter if it performs well, but I would like to
know that is it possible. With enough time it would then be possible
to create an optimized neural network which performed well at the
task. That being said, I truly do hope I can implement a
well-functioning neural network as I think it would be very
interesting to see.

\subsection*{Non-Functional Specification}

In order to complete this project, access to robots will be necessary
to verify results gathered in simulation. After data has been
gathered in simulation, a few data points can be collected on real
robots, although this is not so important since it is the messages
passing through the system which are of interest. Other than that, no
additional materials need to be provided by the school. All the
software required is freely available.

\section*{Approach}

I am taking 70 credits in the first term as well as applying for PhD
positions in the US, so my output will be less than the second
term. My goals for the first term are to have implemented the
connection to both middlewares and to have collect enough data to
analyze the efficacy of the probabilistic model. Also, since I am
taking the neural computation module, I will use this term to get a
strong enough background in neural networks to be able to implement
them.

Second term will be markedly lighter. I will use this time to
implement the neural network for the data-driven technique. Because it
can be difficult to create neural networks, I will start with the
simplest kind appropriate for the task. As I get results, I will try
to make improvements on the neural network architecture. This approach
means I will always have a working system which can classify the
system state. As I get working neural networks, I will be able to
collect results to determine their efficacy.

Additionally, towards the end of second term, I will create a
presentation. I will have to determine what is the best approach for
this. Fault-detection is inherently not very visual, so I may present
graphs of the system state over time and run a quick example of a
robot having a fault in simulation. This is idea will become more
refined as the presentation approaches.

During the Spring holiday, I plan to finish the project. I will write
my report from the detailed notes I plan to take. This technique
worked very well for team Java and it allowed me to create a nice
report.
\end{document}

%%% Local Variables:
%%% TeX-engine: xetex
%%% End:

%  LocalWords:  middleware Hawes xelatex middlewares ROS
