% !TEX TS-program = xelatex
% !TEX encoding = UTF-8 Unicode
\documentclass[10pt, a4paper]{article}
\usepackage{fontspec}

% Document data
\title{Final Year Project Proposal:\\
  Data-Driven Self-Awareness Model for Robotics Systems}
\date{7 October 2011} 
\author{Jeremiah M. Via}

% Set up fonts
\usepackage{xunicode}
\usepackage{xltxtra}
\defaultfontfeatures{Mapping=tex-text}
\setromanfont [Ligatures={Common}, Numbers={OldStyle}]{Linux Libertine O}


% Layout
\usepackage[xetex]{geometry}
\geometry{a4paper, textheight=8.5in}

% ------------------------------------------------------------------ %
% BEGIN DOCUMENT                                                     %
% ------------------------------------------------------------------ %

\begin{document}
\maketitle

\begin{abstract}
  Fault-detection in robotics systems is a difficult task. Robotics
  systems are continually becoming more complex and thus subtle errors
  are hard to diagnose. Traditional approaches have relied on
  explicitly modeling component faults in code, but this technique
  does not scale to complex robots operating in dynamic
  environments. A new technique involves making the robot self-aware
  to the internal state of its various components. The aim of this
  project is to measure the efficacy of a probabilistic self-awareness
  model for robotics systems on two robotics middlewares, ROS and
  CAST, and if time allows, extend the model using a neural network,
  and then compare the two approaches. During this project, I will be
  collaborating with the creator of the original approach.
\end{abstract}


\section*{Description}

This is a project with a heavy research bias. It involves implementing
software to connect the CAST and ROS robotics middlewares to
state-of-the-art software for data-driven fault detection and then
measuring how well this fault-detection system performs on these
middlewares. It also involves extending this technique with a
different machine learning approach, neural networks. My hypothesis is
that neural networks should generalize better than the current
approach.

This software has multiple uses. It can be used during development as
a debugging tool for the developer. They can see if their software is
causing the robot to enter faulty states casued by any number of
things. More interestingly, this information can be fed back to the
robot's planning system. This is what truly creates that idea of
self-awareness. A robot which can determine if its components are not
running correctly can take action to overcome these faults, and is
thus much more robust. This is of critical importance when robots are
working around humans.

\subsection*{User Specification}
% – User specification (intended user(s))

This software is intended for robotics researchers and possibly those
developing industrial robots. It will require a minimal understanding
of machine learning to successfully incorporate into a robotics
system. As such, this software will target expert computer users at
the cutting-edge of robotics. I hope to make the software as intuitive
as possible

\subsection*{Functional Specification}
% – Brief functional specification (what software will do)
% • Essential
% • Optional (if time allows)

In order to measure the efficacy of the probabilistic self-awareness
model, a brdge will have to be written to connect the two middlewares
to the system which implements the probabilistic model. This is
absolutely essential.


\subsection*{Non-Functional Specification}
% – Brief non-functional specification (mandatory software,
% hardware, databases, speed, network access, etc)
% – Resources (software, hardware, information, databases, ...)

In order to complete this project, acess to robots will be necessary
to verify results gathered in simulation.

\section*{Approach}
% should be ambitous and provide several fall-back positions but with
% substantial and self-contained core

% Project plan, in weeks, considering: information gathering and
% learning new skills, specification, design, implementation,
% software testing, user testing, evaluation, etc; do not forget to
% include preparation for project inspection and project
% presentation, write-up and holiday (!)

I am taking 70 credits in the first term as well as applying for PhD
positions in the US, so my output will be less than the second
term. My goals for the first term are to have implemented the
connection to both middlewares and to have collect enough data to
analyze the efficacy of the probabilistic model. Also, since I am
taking the neural computation module, I will use this term to get a
strong enough background in neural networks to be able to implement
them.

Second term will be markedly lighter. I will use this time to
implement the neural network for the data-driven technique. Because it
can be difficult to create neural networks, I will start with the
simplest kind appropriate for the task. As I get results, I will try
to make improvements on the neural network architecture. This approach
means I will always have a working system which can classify the
system state. As I get working neural networks, I will be able to
collect results to determine their efficacy.

Additionally, towards the end of second term, I will create a
presentation. I will have to determine what is the best appraoch for
this. Fault-detection is inherently not very visual, so I may present
graphs of the system state over time and run a quick example of a
robot having a fault in simulation. This is idea will become more
refined as the presentation approaches.

During the Spring holiday, I plan to wrap up my project. I will write
up my report

\end{document}

