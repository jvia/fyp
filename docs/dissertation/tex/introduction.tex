% In this section, you should describe the motivation for the
% project. Explain whatever background the reader will need in order
% to understand your contribution. Include a clear and detailed
% statement of the project aims. Describe the requirement that your
% software is fulfilling. Explain exactly what your software does;
% what input or data it requires, and what output or results it
% delivers, or how it interacts with the user. Concentrate on what
% your software does; the question of how it does it will be dealt
% with in later sections.
%
% The introduction is the most important section of the document;
% everyone who reads your dissertation will read the introduction, and
% many will read only that section. You have to make it as inviting
% and compelling as you can. Think carefully about what you want to
% say, and in what order you should say it. As well as being the most
% important section, it's also the most difficult one to write,
% because it is hard to balance the requirements of giving adequate
% explanation without entering into too much detail.
%
% For these reasons, you should write the abstract and introduction
% last. It is only when you've written the rest of the dissertation
% that you know what you have to introduce.
%
% Conventionally, the last part of the introduction outlines the
% remainder of the dissertation, explaining what comes in each
% section.
\chapter{Introduction}
\label{chap:introduction}
Attention grabbing intro!

% - Lay out a grand vision and state how my project fits within it.
% 
% - Treat the subject as if you're going to keep developing it.
%
% - Give basic idea of project

\begin{itemize}
\item Explain project motivation
\item Explain background needed to understand my work
\item Clear \& detailed statement of project aims
\item Describe requirements of project
\item Explain exactly what your software does; 
  \begin{itemize}
  \item what input or data it requires
  \item what output or results it delivers
  \item how it interacts with the user.
  \item concentrate on what your software does; the question of how it
    does it will be dealt with in later sections.
  \end{itemize}
\end{itemize}


%%% Local Variables:
%%% TeX-engine: xetex
%%% TeX-master: "../master"
%%% End:
