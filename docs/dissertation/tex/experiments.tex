\chapter{Experiments}
\label{chap:experiments}

Running a carefully controlled series of experiments was critical to
determining the efficacy of the various models covered in
chapter~\ref{chap:implementation}. The first six experiments compare
each of the models on a variety of systems. The following experiment
details how ROC analysis can be performed against a model in order to
find the best parameters for any observation input.

\section{Linear system experiment}
\subsection{Aim}
This experiment compared the four models against one another on a
simple linear system with no branching. This was a useful experiment
to determine if the basic premise of any alternations to the learned
model could work in a real system.

\subsection{Method}
\subsection{Results}

\section{Complex system experiment}
\subsection{Aim}
Success on the previous experiment meant that the learned model could
be tried on a more complex system. The aim of this experiment was to
determine if the model could handle more complex interactions between
components. The test was to see if the model could detect systems
that, for example, had components go offline while the majority of the
system still functioned. These subtler errors would be harder to
detect.

\subsection{Method}
\subsection{Results}

\section{Non-connected component experiment}
\subsection{Aim}
The aim of this experiment was to compare the models on a system of
non-connected components. Each of the four models presented in chapter
\ref{chap:implementation} would represent this kind of relationship in
different ways.

\subsection{Method}
\subsection{Results}

\section{Multiple linear systems experiment}
\subsection{Aim}
This experiment tested the various models on a system containing a
multiple non-connected instances of a linear system. This is more
representative of an actual robotics system, where multiple subsystems
are each responsible for their own tasks independent of the other
subsystems.

\subsection{Method}
\subsection{Results}

\section{Dora experiment}
\subsection{Aim}
The aim of this experiment was to test the various models on a real
robotics system. While the other experiments stretched the algorithm
in various ways, this experiment would see how the algorithm fared on
a real system.

\subsection{Method}
\subsection{Results}

\section{ROC analysis on metronome model}
\subsection{Aim}
\subsection{Method}
\subsection{Results}

\section{Detecting faults in extremely large systems}
\subsection{Aim}
\subsection{Method}
\subsection{Results}

%%% Local Variables: 
%%% TeX-engine: xetex
%%% TeX-master: "../master"
%%% End: 
