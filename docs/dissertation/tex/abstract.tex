% The Abstract should be a succinct and self-standing summary of the
% basis and achievements of the project. Be straightforward and
% factual and avoid vague statements, confusing details and "hype". Do
% not be tempted to use acronyms or jargon to keep within the
% half-page limit. Consider that search engines, librarians and
% non-computer scientists wishing to classify your dissertation rely
% on the abstract. You may if you wish provide a short list of
% keywords (2-6 is reasonable) at the end of the abstract.
\begin{abstract}
  Fault-detection in robotics systems is a difficult task and as
  systems are becoming more larger and complex, subtle errors are
  becoming harder to diagnose. Traditional fault-detection approaches
  have relied on explicit modeling of component behavior, but this
  technique does not scale to complex robots operating in dynamic
  environments. A new technique which involves making the robot
  self-aware to the internal state of its various components is
  examined. The aim of this project is to implement and then measure
  the efficacy of this probabilistic self-awareness model for the
  robotics middleware CAST, and if time allows, deal with shortcomings
  of the original approach.

  \vspace{0.5cm}\noindent\textit{Keywords}: robotics, fault detection,
  machine learning
\end{abstract}

%%% Local Variables:
%%% TeX-engine: xetex
%%% TeX-master: "../master"
%%% End: