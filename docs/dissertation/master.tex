%% !TEX TS-program = xelatex
%% !TEX encoding = UTF-8 Unicode
\documentclass[11pt, a4paper]{report}
% Set up fonts
\usepackage{fontspec}
\usepackage{xunicode}
\usepackage{xltxtra}
\defaultfontfeatures{Mapping=tex-text}
\setromanfont[Ligatures={Common}, Numbers={OldStyle}]{Linux Libertine O}
% Layout
\usepackage[xetex]{geometry}
\usepackage{geometry}
\geometry{a4paper, textheight=8.5in}
% title page stuff
\usepackage{setspace}
\usepackage{graphicx}
\usepackage{mathtools}

% ------------------------------------------------------------------ %
% BEGIN DOCUMENT                                                     %
% ------------------------------------------------------------------ %
\begin{document}
\begin{titlepage}
  %% Set the line spacing to 1 for the title page.
  \begin{spacing}{1}
    \begin{large}
      \begin{center}
        \mbox{}
        \vfill
        \begin{sc}
          A Data-Driven Self-Awareness Model for Robotics Systems \\
        \end{sc}
        \vfill
        Jeremiah M. Via \\
        Supervisor: Nick Hawes \\
        \vspace*{4mm}
        \includegraphics[width=50mm]{crest.png}\\
        Submitted in conformity with the requirements\\
        for the degree of Artificial Intelligence \& Computer Science\\
        School of Computer Science\\
        University of Birmingham\\
        \vfill
        Copyright {\copyright} 2012 School of Computer Science, University of Birmingham\\
        \vspace*{.2in}
      \end{center}
    \end{large}
  \end{spacing}
\end{titlepage}

\tableofcontents
%%%%%%%%%%%%%%%%%%%%%%%%%%%%%%%%%%%%%%%%%%%%%%%%%%%%%%%%%%%%%%%%%%%%%%
%% Abstract
%%%%%%%%%%%%%%%%%%%%%%%%%%%%%%%%%%%%%%%%%%%%%%%%%%%%%%%%%%%%%%%%%%%%%%
\begin{abstract}
  Fault-detection in robotics systems is a difficult task and as
  systems are becoming more larger and complex, subtle errors are
  becoming harder to diagnose. Traditional fault-detection approaches
  have relied on explicit modeling of component behavior, but this
  technique does not scale to complex robots operating in dynamic
  environments. A new technique which involves making the robot
  self-aware to the internal state of its various components is
  examined. The aim of this project is to implement and then measure
  the efficacy of this probabilistic self-awareness model for the
  robotics middleware CAST, and if time allows, deal with shortcomings
  of the original approach.

  \vspace{0.5cm}
  \noindent\textit{Keywords}: robotics, fault detection,
  machine learning
\end{abstract}

%%%%%%%%%%%%%%%%%%%%%%%%%%%%%%%%%%%%%%%%%%%%%%%%%%%%%%%%%%%%%%%%%%%%%%
%% Acknowledgments
%%%%%%%%%%%%%%%%%%%%%%%%%%%%%%%%%%%%%%%%%%%%%%%%%%%%%%%%%%%%%%%%%%%%%%
\renewcommand{\abstractname}{Acknowledgments}
\begin{abstract}
  Thanks Mum!
\end{abstract}

%%%%%%%%%%%%%%%%%%%%%%%%%%%%%%%%%%%%%%%%%%%%%%%%%%%%%%%%%%%%%%%%%%%%%%
%% Motivation
%%%%%%%%%%%%%%%%%%%%%%%%%%%%%%%%%%%%%%%%%%%%%%%%%%%%%%%%%%%%%%%%%%%%%%
\chapter{Motivation}
\label{cha:motivation}

\begin{itemize}
\item What is the problem?
\item Robots works with people more
\item Because of this, fault detection is becoming more important. It
  would be terrible for a fault to cause a robot to injure a human.
\item What was my my base goal for the project?
\item Introduce the rest of the dissertation
\item intro literature review
\item intro background
\item intro mention reducing the model
\item intro project management
\item intro project evaluation
\end{itemize}

Something \cite{Antonelo2008}.
%%%%%%%%%%%%%%%%%%%%%%%%%%%%%%%%%%%%%%%%%%%%%%%%%%%%%%%%%%%%%%%%%%%%%%
%% Literature review
%%%%%%%%%%%%%%%%%%%%%%%%%%%%%%%%%%%%%%%%%%%%%%%%%%%%%%%%%%%%%%%%%%%%%%
\chapter{Literature Review}
\label{cha:lit-review}

\begin{itemize}
\item Who else has wanted to solve it and how did they do it?
\item Two basic approaches: model-based \& data driven
\item Model base:
  \begin{itemize}
  \item follows an analytical or knowledge-based approach
  \item model state beforehand and use this to estimate model to
    estimate current system state
  \item precise and fairly close to the hardware
  \item very time-consuming so less interest in complex cognitive
    systems
  \end{itemize}
\end{itemize}

\section{Model-based Approaches}
\section{Data-driven Approaches}

%%%%%%%%%%%%%%%%%%%%%%%%%%%%%%%%%%%%%%%%%%%%%%%%%%%%%%%%%%%%%%%%%%%%%%
%% Background
%%%%%%%%%%%%%%%%%%%%%%%%%%%%%%%%%%%%%%%%%%%%%%%%%%%%%%%%%%%%%%%%%%%%%%
\chapter{Background}\label{cha:background}

\begin{itemize}
\item Give the necessary background to understand Raphael's approach.
\item Explain the math concisely.
\item Give detailed explanation of the learned model so later
  asymptotic analysis fits the story.
\item Include a 2 component system which will serve as the example
  throughout the paper.
\item Use the example system to create an example matrix in all its
  glory
\item Mention CAST and give a brief overview of how it works, perhaps
  juxtaposing it to XCF (although this may be out of the dissertation
  scope).
\end{itemize}

\section{Learning a model}

% Example system:
% a      b      c
% A ---> B ---> C --->

The trained model:
\[
\begin{bmatrix}
  P_{aa} & P_{ab} & P_{ac}\\
  P_{ba} & P_{bb} & P_{bc}\\
  P_{ca} & P_{cb} & P_{cc}
\end{bmatrix}
\]

% Another example system
% a'     b
% A +---> B -->
% | a''    c
% +---> C -->
\[
\begin{bmatrix}
  P_{a'a'}  & P_{a'a''}  & P_{a'b}  & P_{a'c}\\
  P_{a''a'} & P_{a''a''} & P_{a''b} & P_{a''c}\\
  P_{ba'}   & P_{ba''}   & P_{bb}   & P_{bc}\\
  P_{ca'}   & P_{ca''}   & P_{cb}   &  P_{cc}
\end{bmatrix}
\]

\section{Score Calculation}
\section{System Classification}
\section{CAST}

% Train a model for each tuple $(e_i,e_j)$ in the Cartesian product of
% the unique set of observations $U_E$.

% The model of the system then becomes $M = \{P_{ij}|(e_i,e_j) \in U_E
% \times U_E$

%%%%%%%%%%%%%%%%%%%%%%%%%%%%%%%%%%%%%%%%%%%%%%%%%%%%%%%%%%%%%%%%%%%%%%
%% Original system
%%%%%%%%%%%%%%%%%%%%%%%%%%%%%%%%%%%%%%%%%%%%%%%%%%%%%%%%%%%%%%%%%%%%%%
\chapter{Original Approach}
\label{cha:orig-approach}

\begin{itemize}
\item Original system: the idea (just mention because it should be
  explained in the background), the implementation, and the
  experimental results.
\item Refer back to the original idea explain in the background
\item Talk about implementation (FTS, CAST component, modifying aucom, etc.)
\item Experimental setups
\item Experimental results
\item Asymptotic analysis as motivation for my extension.
\end{itemize}

\section{Implementation}
\subsection{FTS Graph processor}
\subsection{CAST component}
\section{Experiments}
\subsection{Setups}
\subsection{Results}
\section{Asymptotic analysis}

%%%%%%%%%%%%%%%%%%%%%%%%%%%%%%%%%%%%%%%%%%%%%%%%%%%%%%%%%%%%%%%%%%%%%%
%% Reduced system
%%%%%%%%%%%%%%%%%%%%%%%%%%%%%%%%%%%%%%%%%%%%%%%%%%%%%%%%%%%%%%%%%%%%%%
\chapter{Connection-based Model}
\label{cha:conn-based-model}

\[
\begin{bmatrix}
  P_{aa} & \empty & \empty\\
  P_{ba} & P_{bb} & \empty\\
  \empty & P_{cb} & P_{cc}
\end{bmatrix}
\]

\begin{itemize}
\item First attempt, using a connection-based model
\item the connection-based idea
\item refer to jen's work
\item implementation
\item asymptotic analysis of the model
\item reminder about experimental setups
\item experimental results
\end{itemize}

\section{Idea}
\section{Asymptotic analysis}
\section{Implementation}
\section{Experiments}

%%%%%%%%%%%%%%%%%%%%%%%%%%%%%%%%%%%%%%%%%%%%%%%%%%%%%%%%%%%%%%%%%%%%%%
%% Metronome system
%%%%%%%%%%%%%%%%%%%%%%%%%%%%%%%%%%%%%%%%%%%%%%%%%%%%%%%%%%%%%%%%%%%%%%
\chapter{Metronome Model}
\label{cha:metronome-model}

\begin{itemize}
\item Metronome-based approach:
\item the idea (showing a simple worked
  example)
\item the implementation
\item asymptotic analysis.
\item remind experiment setup
\item Show experimental results showing that this approach is just as
  performant as the original system.
\item Maybe worth showing the ROC analysis of the three systems to show
  how they perform under a range of parameter values.
\end{itemize}

\section{Idea}

% Metronome model
%
% a      b      c
% A ---> B ---> C --->
%
% M
\[
\begin{bmatrix}
  P_{am} & P_{bm} & P_{cm} & P_{mm}
\end{bmatrix}
\]

\section{Asymptotic analysis}
\section{Implementation}
\section{Experiments}
\subsection{ROC analysis}

%%%%%%%%%%%%%%%%%%%%%%%%%%%%%%%%%%%%%%%%%%%%%%%%%%%%%%%%%%%%%%%%%%%%%%
%% Management
%%%%%%%%%%%%%%%%%%%%%%%%%%%%%%%%%%%%%%%%%%%%%%%%%%%%%%%%%%%%%%%%%%%%%%
\chapter{Project Management}
\label{cha:project-management}
Large projects are strenuous. Effective project management then
becomes crucial in ensuring constant progress throughout all periods
of the academic year.

Git was used rather than Subversion for one key reason: it is easy to
maintain multiple branches of the code and move changes to all of
them. This feature was especially important because it meant that
multiple ideas about the model implementation could be kept in
separate branches. In Subversion, doing the equivalent would have made
it very difficult to make updates to all branches when bugs were found
and fixed.

Because inheriting such a large code-base can be overwhelming, unit
tests were used to create a contract of behavior for the most critical
classes in the system. And by using Jenkins as a continuous
integration server, it was possible to know when any change to the
code caused a test on any branch to fail. Jenkins also published the
results of static analysis run by Maven, the build system used. Static
analysis helped suss out potential bugs and resulted in more robust code.

Perhaps the most important aspect of project management, and
unfortunately discovered only towards the end of the project, was
issue management. It was possible to set project milestones and attach
the issues necessary to complete the milestone. This has the benefit
of putting in concrete terms the steps necessary to reach a goal. So
rather than flailing around to figure out what to do next, there was
always a concrete task that could be done.

%%%%%%%%%%%%%%%%%%%%%%%%%%%%%%%%%%%%%%%%%%%%%%%%%%%%%%%%%%%%%%%%%%%%%%
%% Project evaluation
%%%%%%%%%%%%%%%%%%%%%%%%%%%%%%%%%%%%%%%%%%%%%%%%%%%%%%%%%%%%%%%%%%%%%%
\chapter{Project Evaluation}
\label{cha:project-evaluation}

\begin{itemize}
\item Honest project evaluation:
\item Talk about what I did right
  \begin{itemize}
  \item summer work in anticipation of a busy first term
  \end{itemize}
\item the things from which I can learn.
  \begin{itemize}
  \item sticking with something even when it doesn't work
  \item setting better goals
  \end{itemize}
\end{itemize}

%%%%%%%%%%%%%%%%%%%%%%%%%%%%%%%%%%%%%%%%%%%%%%%%%%%%%%%%%%%%%%%%%%%%%%
%% Conclusion
%%%%%%%%%%%%%%%%%%%%%%%%%%%%%%%%%%%%%%%%%%%%%%%%%%%%%%%%%%%%%%%%%%%%%%
\chapter{Conclusion \& Future Work}
\label{cha:conclusion}

\begin{itemize}
\item Tie the story together \& mention some ideas for future
  work.
\item Multiple-models
\item Learning to detect specific faults
\end{itemize}


% 
\begin{abstract}
  This is pretty abstract.
\end{abstract}

%%% Local Variables:
%%% TeX-master: "master"
%%% End:
% \renewcommand{\abstractname}{Acknowledgements}
\begin{abstract}
 Thanks Mum!
\end{abstract}

%%% Local Variables: 
%%% TeX-engine: xetex
%%% TeX-master: "../master"
%%% End: 

% % In this section, you should describe the motivation for the
% project. Explain whatever background the reader will need in order
% to understand your contribution. Include a clear and detailed
% statement of the project aims. Describe the requirement that your
% software is fulfilling. Explain exactly what your software does;
% what input or data it requires, and what output or results it
% delivers, or how it interacts with the user. Concentrate on what
% your software does; the question of how it does it will be dealt
% with in later sections.
%
% The introduction is the most important section of the document;
% everyone who reads your dissertation will read the introduction, and
% many will read only that section. You have to make it as inviting
% and compelling as you can. Think carefully about what you want to
% say, and in what order you should say it. As well as being the most
% important section, it's also the most difficult one to write,
% because it is hard to balance the requirements of giving adequate
% explanation without entering into too much detail.
%
% For these reasons, you should write the abstract and introduction
% last. It is only when you've written the rest of the dissertation
% that you know what you have to introduce.
%
% Conventionally, the last part of the introduction outlines the
% remainder of the dissertation, explaining what comes in each
% section.
\chapter{Introduction}
\label{chap:introduction}

Introduce stuff here.

%%% Local Variables:
%%% TeX-engine: xetex
%%% TeX-master: "../master"
%%% End:
 %% Motivation & introduction for rest of report
% \chapter{Background}
\label{chap:background}

\begin{itemize}
\item Literature review
\item Detailed aucom explanation
\end{itemize}


%%% Local Variables:
%%% TeX-engine: xetex
%%% TeX-master: "../master"
%%% End:
   %% Lit. review & aucom background
% \chapter{Original Implementation}
%%% Local Variables:
%%% TeX-engine: xetex
%%% TeX-master: "../master"
%%% End:

% \include{tex/connection}
% \include{tex/metronome}
% % \chapter{Experiments}
\label{chap:experiments}

\section{Three chain experiment}
\section{Five chain experiment}
\section{Individual component experiment}
\section{Parallel experiment}


%%% Local Variables: 
%%% TeX-engine: xetex
%%% TeX-master: "../master"
%%% End: 

% \chapter{Management}
\label{chap:management}


\begin{itemize}
  \item git
  \item issues
  \item jenkins
  \item unit testing
\end{itemize}


%%% Local Variables: 
%%% TeX-engine: xetex
%%% TeX-master: "../master"
%%% End: 

% \chapter{Evaluation}
\label{chap:evaluation}


\begin{itemize}
  \item How well did the project achieve what you had hoped for?
  \item Be frank!
  \item No waffle
  \item Would people want to use it?
  \item Don't be too negative
\end{itemize}


%%% Local Variables: 
%%% TeX-engine: xetex
%%% TeX-master: "../master"
%%% End: 

% \chapter{Conclusion}
\label{chap:conclusion}

%%% Local Variables: 
%%% TeX-engine: xetex
%%% TeX-master: "../master"
%%% End: 

\bibliographystyle{plain}
\bibliography{references.bib}
\end{document}

%%% Local Variables:
%%% TeX-engine: xetex
%%% End:





% LocalWords:  tuple analytical
