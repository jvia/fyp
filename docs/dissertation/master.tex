%% !TEX TS-program = xelatex
%% !TEX encoding = UTF-8 Unicode
%%\documentclass[11pt, a4paper]{report}
\documentclass[11pt,a4paper]{article}
% Set up fonts
\usepackage{fontspec}
\usepackage{xunicode}
\usepackage{xltxtra}
\defaultfontfeatures{Mapping=tex-text}
\setromanfont[Ligatures={Common}, Numbers={OldStyle}]{Linux Libertine O}
% Layout
\usepackage[xetex]{geometry}
\usepackage{geometry}
\geometry{a4paper, textheight=8.5in}
\usepackage{setspace}
\usepackage{wrapfig}
\usepackage{graphicx}
\graphicspath{{./diagrams}{./images}}
\DeclareGraphicsExtensions{.eps,.ps,.pdf,.png}
\usepackage{mathtools}
\usepackage{paralist}
% ------------------------------------------------------------------ %
% BEGIN DOCUMENT                                                     %
% ------------------------------------------------------------------ %
\begin{document}
\begin{titlepage}
  %% Set the line spacing to 1 for the title page.
  \begin{spacing}{1}
    \begin{large}
      \begin{center}
        \mbox{}
        \vfill
        \begin{sc}
          A Data-Driven Self-Awareness Model for Robotics Systems \\
        \end{sc}
        \vfill
        Jeremiah M. Via \\
        Supervisor: Nick Hawes \\
        \vspace*{4mm}
        \includegraphics[width=50mm]{crest.png}\\
        Submitted in conformity with the requirements\\
        for the degree of Artificial Intelligence \& Computer Science\\
        School of Computer Science\\
        University of Birmingham\\
        \vfill
        Copyright {\copyright} 2012 School of Computer Science, University of Birmingham\\
        \vspace*{.2in}
      \end{center}
    \end{large}
  \end{spacing}
\end{titlepage}

\tableofcontents
\newpage
%%%%%%%%%%%%%%%%%%%%%%%%%%%%%%%%%%%%%%%%%%%%%%%%%%%%%%%%%%%%%%%%%%%%%%
%% Abstract
%%%%%%%%%%%%%%%%%%%%%%%%%%%%%%%%%%%%%%%%%%%%%%%%%%%%%%%%%%%%%%%%%%%%%%
\begin{abstract}
  Fault-detection in robotics systems is a difficult task and as
  systems are becoming more larger and complex, subtle errors are
  becoming harder to diagnose. Traditional fault-detection approaches
  have relied on explicit modeling of component behavior, but this
  technique does not scale to complex robots operating in dynamic
  environments. A new technique which involves making the robot
  self-aware to the internal state of its various components is
  examined. The aim of this project is to implement and then measure
  the efficacy of this probabilistic self-awareness model for the
  robotics middleware CAST, and if time allows, deal with shortcomings
  of the original approach.

  \vspace{0.5cm}
  \noindent\textit{Keywords}: robotics, fault detection,
  machine learning
\end{abstract}
\newpage

%%%%%%%%%%%%%%%%%%%%%%%%%%%%%%%%%%%%%%%%%%%%%%%%%%%%%%%%%%%%%%%%%%%%%%
%% Acknowledgments
%%%%%%%%%%%%%%%%%%%%%%%%%%%%%%%%%%%%%%%%%%%%%%%%%%%%%%%%%%%%%%%%%%%%%%
\renewcommand{\abstractname}{Acknowledgments}
\begin{abstract}
  Thanks Mum!
\end{abstract}
\newpage

%%%%%%%%%%%%%%%%%%%%%%%%%%%%%%%%%%%%%%%%%%%%%%%%%%%%%%%%%%%%%%%%%%%%%%
%% Motivation
%%%%%%%%%%%%%%%%%%%%%%%%%%%%%%%%%%%%%%%%%%%%%%%%%%%%%%%%%%%%%%%%%%%%%%
\section{Motivation}
\label{cha:motivation}
The number of robots within in the economy is increasing as are their
range of tasks. There are robots in factories automating difficult or
repetitive tasks. There are an increasing number of domestic robots
being used for assistance in the home or as entertainment. It is
important for these robots to function correctly, and if unable to do
so, to degrade gracefully to minimize harm to themselves and others.
To do this, robots need some way to determine their own operating
conditions. Detecting faults within robot systems is a hard problem.

The importance of equipping a robot with the ability of self-awareness
to its internal state increases as humans interact more with robots.
One could imagine a robot in the home which assisted an elderly person
or \textit{another situation}. A malfunction in these situations a
malfunction could cause serious harm to a human. One could also
imagine robotic arms at a factory building cars. A malfunction could
cause damage to products or the arm itself resulting in a loss of
factory output. These examples underscore the importance of detecting
and handling faults within a robot system.

The main goal of this project was to determine if the technique
developed by Raphael Golombek \cite{Golombek2011} could work on the
CoSY Architecture Schema Toolkit (CAST) \cite{haweswyatt10aei}. To do
this, Raphael's software was extensively modified to work with CAST
and then experiments were performed to determine the efficacy of this
approach. The extension for this project was to find a way to deal
with the size of learned model so that this technique could scale to
large systems.

\noindent What was my my base goal for the project?
%% - Modify Raph's system to work with CAST
%% - examine the efficacy of this system
%% - original extension: use a NN for detection
%% - actual extension: increase the efficient of the learned model

The rest of this dissertation is composed of the following main
sections:
\begin{inparaenum}[\itshape a\upshape)]
\item a brief literature review;
\item an explanation of the technique;
\item analysis of the technique and potential improvements;
\item and a project evaluation.
\end{inparaenum} The literature review will briefly explain approaches
taken by others attempting to solve the problem previously mentioned.
This will give the reader context from which to understand the work.


%%%%%%%%%%%%%%%%%%%%%%%%%%%%%%%%%%%%%%%%%%%%%%%%%%%%%%%%%%%%%%%%%%%%%%
%% Literature review
%%%%%%%%%%%%%%%%%%%%%%%%%%%%%%%%%%%%%%%%%%%%%%%%%%%%%%%%%%%%%%%%%%%%%%
\section{Literature Review}
\label{cha:lit-review}

\begin{itemize}
\item Who else has wanted to solve it and how did they do it?
\item Two basic approaches: model-based \& data driven
\item Model base:
  \begin{itemize}
  \item follows an analytical or knowledge-based approach
  \item model state beforehand and use this to estimate model to
    estimate current system state
  \item precise and fairly close to the hardware
  \item very time-consuming so less interest in complex cognitive
    systems
  \end{itemize}
\end{itemize}

\subsection{Model-based Approaches}
\subsection{Data-driven Approaches}

%%%%%%%%%%%%%%%%%%%%%%%%%%%%%%%%%%%%%%%%%%%%%%%%%%%%%%%%%%%%%%%%%%%%%%
%% Background
%%%%%%%%%%%%%%%%%%%%%%%%%%%%%%%%%%%%%%%%%%%%%%%%%%%%%%%%%%%%%%%%%%%%%%
\section{Background}\label{cha:background}

\begin{itemize}
\item Give the necessary background to understand Raphael's approach.
\item Explain the math concisely.
\item Give detailed explanation of the learned model so later
  asymptotic analysis fits the story.
\item Include a 2 component system which will serve as the example
  throughout the paper.
\item Use the example system to create an example matrix in all its
  glory
\item Mention CAST and give a brief overview of how it works, perhaps
  juxtaposing it to XCF (although this may be out of the dissertation
  scope).
\end{itemize}

\subsection{Learning a model}


\begin{wrapfigure}{r}{0.5\textwidth}
  \includegraphics[width=0.48\textwidth]{diagrams/ex1.pdf}
  \caption{example1}
  \label{fig:ex1}
\end{wrapfigure}

The trained model:
\[
\begin{bmatrix}
  P_{aa} & P_{ab} & P_{ac}\\
  P_{ba} & P_{bb} & P_{bc}\\
  P_{ca} & P_{cb} & P_{cc}
\end{bmatrix}
\]

\begin{wrapfigure}{r}{0.5\textwidth}
  \includegraphics[width=0.48\textwidth]{diagrams/ex2.pdf}
  \caption{example2}
  \label{fig:ex2}
\end{wrapfigure}
\[
\begin{bmatrix}
  P_{a'a'}  & P_{a'a''}  & P_{a'b}  & P_{a'c}\\
  P_{a''a'} & P_{a''a''} & P_{a''b} & P_{a''c}\\
  P_{ba'}   & P_{ba''}   & P_{bb}   & P_{bc}\\
  P_{ca'}   & P_{ca''}   & P_{cb}   &  P_{cc}
\end{bmatrix}
\]

\subsection{Score Calculation}
\subsection{System Classification}
\subsection{CAST}

% Train a model for each tuple $(e_i,e_j)$ in the Cartesian product of
% the unique set of observations $U_E$.

% The model of the system then becomes $M = \{P_{ij}|(e_i,e_j) \in U_E
% \times U_E$

%%%%%%%%%%%%%%%%%%%%%%%%%%%%%%%%%%%%%%%%%%%%%%%%%%%%%%%%%%%%%%%%%%%%%%
%% Original system
%%%%%%%%%%%%%%%%%%%%%%%%%%%%%%%%%%%%%%%%%%%%%%%%%%%%%%%%%%%%%%%%%%%%%%
\section{Original Approach}
\label{cha:orig-approach}

\begin{itemize}
\item Original system: the idea (just mention because it should be
  explained in the background), the implementation, and the
  experimental results.
\item Refer back to the original idea explain in the background
\item Talk about implementation (FTS, CAST component, modifying aucom, etc.)
\item Experimental setups
\item Experimental results
\item Asymptotic analysis as motivation for my extension.
\end{itemize}

\subsection{Implementation}
\subsubsection{FTS Graph processor}
\subsubsection{CAST component}
\subsection{Experiments}
\subsubsection{Setups}
\subsubsection{Results}
\subsection{Asymptotic analysis}

%%%%%%%%%%%%%%%%%%%%%%%%%%%%%%%%%%%%%%%%%%%%%%%%%%%%%%%%%%%%%%%%%%%%%%
%% Reduced system
%%%%%%%%%%%%%%%%%%%%%%%%%%%%%%%%%%%%%%%%%%%%%%%%%%%%%%%%%%%%%%%%%%%%%%
\section{Connection-based Model}
\label{cha:conn-based-model}

\[
\begin{bmatrix}
  P_{aa} & \empty & \empty\\
  P_{ba} & P_{bb} & \empty\\
  \empty & P_{cb} & P_{cc}
\end{bmatrix}
\]

\begin{itemize}
\item First attempt, using a connection-based model
\item the connection-based idea
\item refer to jen's work
\item implementation
\item asymptotic analysis of the model
\item reminder about experimental setups
\item experimental results
\end{itemize}

\subsection{Idea}
\subsection{Asymptotic analysis}
\subsection{Implementation}
\subsection{Experiments}

%%%%%%%%%%%%%%%%%%%%%%%%%%%%%%%%%%%%%%%%%%%%%%%%%%%%%%%%%%%%%%%%%%%%%%
%% Metronome system
%%%%%%%%%%%%%%%%%%%%%%%%%%%%%%%%%%%%%%%%%%%%%%%%%%%%%%%%%%%%%%%%%%%%%%
\section{Metronome Model}
\label{cha:metronome-model}

\begin{itemize}
\item Metronome-based approach:
\item the idea (showing a simple worked
  example)
\item the implementation
\item asymptotic analysis.
\item remind experiment setup
\item Show experimental results showing that this approach is just as
  performant as the original system.
\item Maybe worth showing the ROC analysis of the three systems to show
  how they perform under a range of parameter values.
\end{itemize}

\subsection{Idea}

% Metronome model
%
% a      b      c
% A ---> B ---> C --->
%
% M
\[
\begin{bmatrix}
  P_{am} & P_{bm} & P_{cm} & P_{mm}
\end{bmatrix}
\]

\subsection{Asymptotic analysis}
\subsection{Implementation}
\subsection{Experiments}
\subsubsection{ROC analysis}

%%%%%%%%%%%%%%%%%%%%%%%%%%%%%%%%%%%%%%%%%%%%%%%%%%%%%%%%%%%%%%%%%%%%%%
%% Management
%%%%%%%%%%%%%%%%%%%%%%%%%%%%%%%%%%%%%%%%%%%%%%%%%%%%%%%%%%%%%%%%%%%%%%
\section{Project Management}
\label{cha:project-management}
Large projects are strenuous. Effective project management then
becomes crucial in ensuring constant progress throughout all periods
of the academic year.

Git was used rather than Subversion for one key reason: it is easy to
maintain multiple branches of the code and move changes to all of
them. This feature was especially important because it meant that
multiple ideas about the model implementation could be kept in
separate branches. In Subversion, doing the equivalent would have made
it very difficult to make updates to all branches when bugs were found
and fixed.

Because inheriting such a large code-base can be overwhelming, unit
tests were used to create a contract of behavior for the most critical
classes in the system. And by using Jenkins as a continuous
integration server, it was possible to know when any change to the
code caused a test on any branch to fail. Jenkins also published the
results of static analysis run by Maven, the build system used. Static
analysis helped suss out potential bugs and resulted in more robust code.

Perhaps the most important aspect of project management, and
unfortunately discovered only towards the end of the project, was
issue management. It was possible to set project milestones and attach
the issues necessary to complete the milestone. This has the benefit
of putting in concrete terms the steps necessary to reach a goal. So
rather than flailing around to figure out what to do next, there was
always a concrete task that could be done.

%%%%%%%%%%%%%%%%%%%%%%%%%%%%%%%%%%%%%%%%%%%%%%%%%%%%%%%%%%%%%%%%%%%%%%
%% Project evaluation
%%%%%%%%%%%%%%%%%%%%%%%%%%%%%%%%%%%%%%%%%%%%%%%%%%%%%%%%%%%%%%%%%%%%%%
\section{Project Evaluation}
\label{cha:project-evaluation}

\begin{itemize}
\item Honest project evaluation:
\item Talk about what I did right
  \begin{itemize}
  \item summer work in anticipation of a busy first term
  \end{itemize}
\item the things from which I can learn.
  \begin{itemize}
  \item sticking with something even when it doesn't work
  \item setting better goals
  \end{itemize}
\end{itemize}

%%%%%%%%%%%%%%%%%%%%%%%%%%%%%%%%%%%%%%%%%%%%%%%%%%%%%%%%%%%%%%%%%%%%%%
%% Conclusion
%%%%%%%%%%%%%%%%%%%%%%%%%%%%%%%%%%%%%%%%%%%%%%%%%%%%%%%%%%%%%%%%%%%%%%
\section{Conclusion \& Future Work}
\label{cha:conclusion}

\begin{itemize}
\item Tie the story together \& mention some ideas for future
  work.
\item Multiple-models
\item Learning to detect specific faults
\end{itemize}

%%%%%%%%%%%%%%%%%%%%%%%%%%%%%%%%%%%%%%%%%%%%%%%%%%%%%%%%%%%%%%%%%%%%%%
%% Bibliography
%%%%%%%%%%%%%%%%%%%%%%%%%%%%%%%%%%%%%%%%%%%%%%%%%%%%%%%%%%%%%%%%%%%%%%
\bibliographystyle{plain}
\bibliography{references.bib}
\end{document}

%%% Local Variables:
%%% TeX-engine: xetex
%%% End:
% LocalWords:  tuple analytical Golombek CoSY Raphael's
